\chapter{MATLAB\textsuperscript{\textregistered} implementation}

In this chapter we describe the implementation of the numerical method in MATLAB\textsuperscript{\textregistered}, which was used to perform the experiments described in Chapter \ref{chap:experiments}.

The code is available at the following \href{https://github.com/fofe1125/tesi}{link}. Here we report the main functions and implementation details.

\section{Non uniform grid generation}

The function \texttt{neqdeltagrid0.m} generates a non-uniform grid as described in Section \ref{sb:nufg}. The function takes as input the parameters $\delta$, $h_1$, $L$ and $m$, and returns the grid points in the interval $[0, L]$.

\begin{multicols}{2}
    \begin{lstlisting}

    % from xc to 0
    for i = m-1:-1:2
        h(i) = (1 + delta)*h(i+1);
        x(i) = x(i+1) - h(i);
    end

    % from xc to xL
    i = m + 1;
    while x(i - 1) < xL
        h(i) = (1 + delta)*h(i-1);
        x(i) = x(i-1) + h(i);
        i = i + 1;
    end

    % always add 0
    x = [0, x];
    h = diff(x);

    % check for the nearest point to 0
    if numel(x) > 2
        if h(1) < 0.7*h(2)
            x(2) = [];
            h = diff(x);
        end
    end

    \end{lstlisting}
\end{multicols}

We then construct the full grid in $[-L, L]$ by mirroring the points around zero as

\begin{lstlisting}
x = neqdeltagrid0(xc,Lx, delta, h1);
x = unique([-flip(x), x]); x = x(:);
hx = diff(x);
mx = length(x);
\end{lstlisting}

Once we have the grid points, we can construct the finite difference matrix as 


\begin{lstlisting}
%% matrix
dux = 2./(hx(1:mx-2) .* (hx(1:mx-2) + hx(2:mx-1))); % upper diag 
dmx = -2./(hx(1:mx-2).*hx(2:mx-1));                 % main diag
dlx = 2./(hx(2:mx-1).*(hx(1:mx-2) + hx(2:mx-1)));   % lower diag

Dxx = spdiags([[dux;0;0], [0; dmx; 0], [0;0;dlx]], -1:1, mx,mx);

% Homogeneus Neumann conditions 
Dxx(1,1:2) = [-2,2]/(hx(1)^2);
Dxx(mx,mx-1:mx) = [2,-2]/(hx(mx-1)^2);
\end{lstlisting}

\section{Time splitting}

We report a brief section of the code used to implement the time integration

\begin{lstlisting}
for i = 1:nt - 1

    U = Ex*U*Ey.';
    U = exp(1i*tau/2*(1 - abs(U).^2)).*U;
    U = Ex*U*Ey.';
 
end
\end{lstlisting}